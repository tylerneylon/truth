\documentclass[11pt, oneside]{article}   	% use "amsart" instead of "article" for AMSLaTeX format
\usepackage{geometry}                		% See geometry.pdf to learn the layout options. There are lots.
\geometry{letterpaper}                   		% ... or a4paper or a5paper or ... 
%\geometry{landscape}                		% Activate for rotated page geometry
%\usepackage[parfill]{parskip}    		% Activate to begin paragraphs with an empty line rather than an indent
\usepackage{graphicx}				% Use pdf, png, jpg, or eps§ with pdflatex; use eps in DVI mode
								% TeX will automatically convert eps --> pdf in pdflatex		
\usepackage{amsmath}
\usepackage{amssymb}
\usepackage{float}
\usepackage{hyperref}
\usepackage{wrapfig}
\usepackage{refcount}

\hypersetup{
    colorlinks=true,
    linkcolor=black,   
    urlcolor=cyan,
}

\newtheorem{thm}{Theorem}
\newtheorem{conj}{Conjecture}
\newtheorem{question}{Question}
\newtheorem{idea}{Idea}
\newtheorem{postidea}{Post Idea}
\newtheorem{defn}{Definition}
\newtheorem{ans}{Answer}
\newtheorem{prop}{Property}
\newtheorem{propn}{Proposition}
\newtheorem{claim}{Claim}

\newcommand{\up}[1]{\ensuremath{^\text{#1}}}

\newcommand{\Q}{\mathbb{Q}}
\newcommand{\R}{\mathbb{R}}
\newcommand{\D}{\mathbb{D}}
\newcommand{\Z}{\mathbb{Z}}

\newcommand{\spacer}[1]{\rule[-#1]{0pt}{#1}}

\newcommand{\qed}{\ensuremath{\Box}}
\newcommand\bpf[1][]{\smallskip\noindent{\bf Proof#1.}\quad}
\newcommand\epf{\qed\medskip}
\newcommand\hr{\bigskip\hrule\bigskip}

\begin{document}

{\bf Truth Is Not the Bedrock of Human Knowledge}

Tyler Neylon

\bigskip

This article explores the question:
\begin{question}\label{q1}
    What is truth?
\end{question}

I'll tell you why this question is useful for any explorer of knowledge, and
I'll argue that our concept of truth is a human invention as opposed to an idea
that is intrinsic to the universe.
Moreover, the concept of truth has arbitrary aspects to it, meaning that some of
the things we consider to be true are less elegant and more anthropocentric than
we may first realize. This has profound ramifications on how we, humans, ought
to view both ourselves and the world.

\section{The Box We'll Peek Inside}

A lot of philosophy is poorly motivated, so I'll start by
looking at what useful results can come out of this exploration.

As a kid, I was taught {\em the scientific method} as a way to
learn about the world -- to learn what is true.
I'll roughly summarize the scientific method as the statement
of a hypothesis, along with an alternative to test against, and
the collection and analysis of evidence to distinguish between
the two.

As I grew up, I
learned to question the completeness of this method.
I learned to appreciate the formality of a sound
mathematical proof, for example. A formal proof makes evidence collection seem
crude in
comparison. A logical proof is all-encompassing: the hypothesis -- now a theorem
-- is
precise and final,
invulnerable to the need for later revision.
One pitfall of the scientific method is the realization that it
is a process of guessing, collecting imperfect data, and doing our best to
connect that noisy data within our collection of guesses.

Beyond the awareness that the scientific method is noisy and imperfect, we can
begin to see that some of our questions do not fit cleanly within the confines
of its form. For example, we may find two conflicting theories of physics which
each partially explain the relevant observations. Then we must decide
which theory we believe. One might argue that the holes in one theory are worse
than the holes in another. One might argue that one theory is more elegant than
another.
These conversations aren't well captured by what we traditionally call the
scientific method, yet
they're common occurrences in knowledge-expanding communities,
whether the field is physics, computer science, or even, giving
ourselves some wiggle room, among chefs exploring theories of gastronomical
excellence.

All of this leads to the big question:

\begin{question}\label{q2}
    How can we learn what is true?
\end{question}

Question \ref{q2} can help us evolve how we learn.
It helps us to better understand and to build on what we call our
scientific method.
It may even
be the ultimate practical question -- this is the motivation I promised.

To connect the dots: When you open the door toward answering {\em how can we
learn what is true?}, you find the
Pandora's box that is {\em what is truth?} This is the box we'll open.

\section{Definitions of Truth}

\subsection{Correspondence Truth}

Truth is a concept so fundamental to human thinking that it's elusive to define
in simpler terms.
Perhaps the most traditional approach has been an idea called
{\em correspondence theory}\/:

\begin{defn}[Correspondence Truth]\label{d1}
    An idea is true when it corresponds to reality.
\end{defn}

I don't think this is a good definition.

Pretend we're defining the mathematical idea of a {\em number}, and we said
``a number is an element of the set of numbers.'' This definition of a number
has a lot in common with the correspondence definition of truth:
\begin{itemize}
    \item It relies on another concept ({\em the set of numbers})
        which is more complicated than the thing
        being defined, 
    \item It doesn't add much intuition about the thing being defined, and
    \item It's not an easily testable definition. That is, we don't have a nice
        way to test if a thing is a number or not, in the context of not yet
        knowing a lot about the {\em set of numbers}.
\end{itemize}

On a metalevel, notice that we're doing something interesting here --- we're
looking for the definition of an idea that we already intuitively understand.
How will we know when we have a good definition? Abstractly, when the definition
appeals to our intuition.
Examples of the concept being defined ought to fit well with our
proposed definition.

% XXX Do I want to include this paragraph? It's a little digression from the
%     main line of thought.
I can't help but notice that this process of finding a definition is a case of
knowledge discovery which does not fit well with the traditional scientific
method --- although we could, with some creativity, treat concept examples as
experiments or observations, and consider a definition as a hypothesis.

With that in mind, let's consider an interesting proposition, something that we
think of as a candidate for being true:
\begin{propn}\label{p1}
    If Plato were alive today, he would like pizza more than sushi.
\end{propn}

The correspondence definition of truth fails to shed light
on the truth or falsity of this proposition.
The fact is that Plato is not alive today, and --- for all practical purposes
--- we have no way to determine what kinds of modern food he would most like.
What we could do is make educated guesses, and try to convince each other
that one of these guesses is more likely to be correct.
This method is only loosely connected with an examination of reality.
If we take a careful look at how we
decide something is true or false, then perhaps we can arrive at an improved
definition of truth.

Learning from the weakness of definition \ref{d1}, a good definition:
\begin{itemize}
    \item Relies on prior concepts,
    \item Adds intuition, and
    \item Helps test if something fits the definition.
\end{itemize}

\subsection{Different Kinds of Truths}

When we start to look at {\em how} we decide something is true, we see a number
of different methods. In this section I'll describe kinds of truths classified
in this way. I'm not sure if this list is exhaustive, although it covers all of
the examples I've considered so far.

As a mathematician, I'd like to start with:

\begin{defn}[Mathematical Truth]
    A mathematical idea is true when we can provide a logically sound proof for
    it.
\end{defn}

As a brief reminder, a {\em sound argument} is a series of logical statements in
which the conclusion must logically follow from the premises, and in which the
premises are true in the context of the statement being proven.

The math-loving part of me would like to say that a proven mathematical idea
achieves a kind of ideal level of truthiness. But in practice, this isn't quite
right. There are a number of reasons why mathematical ideas don't typically
achieve ``perfect truth:''
\begin{itemize}
    \item Math relies fundamentally on axioms being true, but those axioms are
        not proven. They are presented as self-evident, and in practice they
        have subtle and non-trivial consequences.
        One example is a geometric axiom called Euclid's parallel postulate,
        which states that, given a line and a point, there is a unique line
        through the point parallel to the line. If we assume this is true, then
        we can arrive at laws of geometry on a flat surface. If we assume this
        is false, then we can arrive at {\em equally valid} laws of geometry on
        non-flat surfaces, such as the surface of a sphere. The point here is to
        show that, as much as we'd like axioms to be self-evident, they aren't
        always so.
    \item Virtually all mathematical proofs are not actually fully formal
        arguments. That is, math literature is written for human consumption,
        and consists of largely natural language persuasion, as opposed to a
        computationally-verifiable proof. In practice, we sometimes find
        mistakes or omissions in these proofs, and they can be quite subtle. For
        example, mathematicians have famously disagreed about the correctness of
        Camille Jordan's original proof of the Jordan curve theorem.
        When experts disagree about the correctness of a proof, it shows that it
        can be quite difficult to see whether a proof is truly correct!
    \item Finally, even the rules of logic themselves are subject to debate. If
        we truly want to assume nothing, then it would be good to {\em know} we
        are using the correct rules of logic, rather than to assume them. This
        is not mere philosophical skepticism --- this is a known loophole in the
        aspired perfection of mathematical reasoning. A proof by contradiction
        only makes sense if we know the axioms themselves do not contradict each
        other --- when we have non-contradicting axioms, they're called {\em
        consistent}. The tricky thing is, mathematicians are unable to prove
        that some important axiom systems are consistent without resorting to
        additional axioms. When we add new axioms, though, we have to prove the
        consistency of the new axioms. And, although this fact is not obvious,
        this process can never end --- we can never {\em prove} the consistency
        of a set of axioms --- when we're talking about number
        systems.\footnote{For the curious: I'm referring to G\"odel's second
        incompleteness theorem as applied to Peano arithmetic.}
\end{itemize}

% TODO State, either before or after this math part, what I'm getting at in the
% big picture. Something like we think these things are true, but we don't have
% as much certainty as we'd like.


\hr

Notes for section 2.

It's like saying a number is an element of the set of numbers.
It hasn't helped us understand anything, and it feels like it's
defined in terms of more complex ideas rather than in terms of
something simpler. It offers us little in the way of testability.

Metanote that we're doing something interesting in finding a philosophical
definition. It's like we're looking for a way to describe something that
we already know, as opposed to a definition in math literature, where, when
learning, the definition comes first (for the student), and the intuition often
follows after.

Segue to the next section? The next section will introduce different types of
truth. Which steak would Einstein have liked better? It's a weird kind of
question because it's hard to test, yet it feels like it still has a true
answer. I'm mentioning this example because I'm trying to show a difference
between corresponding to reality and the way we actually work with certain
claims.

What I'm moving toward here is that I'm aiming for properties that are nice to
have in a good definition.

It's nice if a definition adds intuition to the thing being defined. For
example, can we understand how this concept is useful, or how it historically
evolved. It's nice if the definition gives a way to test if something fits the
definition, and if the definition is built in terms of concepts that we can
understand before we understand the defined idea.

\hr


One way to try on definitions or theories is to see how they work with examples,
so in this article I'll provide examples of candidate truths --- things which
might count as true or false. I'll call these {\em claims}, although I am not
literally claiming these to be true.

\begin{claim}
    The party I'm throwing tonight begins at seven.
\end{claim}





\bigskip
\hrule
\bigskip

{\em MAH NOTES}

Dearest reader, if you see this section at all, then I, the author,
have made some sort of grave error in compiling and/or distributing this
article. I apologize profusely and ensure you that anything untoward said
about you or anyone you care about was surely and entirely in jest.

{\bf Motivation}

I want to be able to figure out if something is true or not.
Now, later on, I'll lean toward the idea of effective truth.
So I will need to say how this ends up being useful, but I'd rather save
that for a recap at the end.

I might want to say how it's not always obvious what's true.
Maybe that can be the first exploration of the article.

{\bf What kind of truths are tricky?}

We think of truth as the state of the world. But often it's not that simple.

For example, did Greedo shoot first? It feels like there should be an answer to
this, but it's not obvious how to decide this.

This sentence is false. Is it?

There are kind of ethical questions.
What makes privacy a fundamental right?
What is bad about talking about taboo subjects?
When we have a candidate answer to these, we want a way to test how true they
are.

Axioms of math. What makes something self-evident?

Newtonian mechanics. We know they it is a false model, but we still find it
useful.

{\bf Defining truth}

Here is an outline of ideas I can cover:

Ideas for different kinds of truth:
\begin{itemize}
    \item Verifiable truth; like a repeatable physics experiment.
    \item Authorative truth; like an author saying what a character did.
    \item Democratic truth; like a group deciding what's cool or polite.
    \item Effective truth; an idea that tends to produce a desired result.
    \item Evolutionary truth; an idea that tends to survive.
          Perhaps it's convincing, or perhaps it is self-reinforcing, like
          religion.
\end{itemize}

\bigskip
\hrule
\bigskip

I may want to take this article in a slightly different direction. I've already
kind of outlined many ideas about the nature of truth in an article called ``A
Model of Human Thought: Philosophy.'' But still, the focus of that article was
not actually truth, and I simply included it because I had not written those
ideas up previously, and I wanted them for a more solid context to model human
thought.

The slightly different direction here could be to build an intuition around the
idea that our concept of truth is a somewhat arbitrary and invented idea. A
practical grounding could be in the basis of wanting to know what is true as a
good scientist. And we get to some quite difficult questions such as, is this
axiom true? Why does anything exist? How do we think about abstract
contradictions such as this-sentence-is-false, or the idea of proof by
contradiction?

And one conclusion is that if we want to expand our boundaries of learning, then
we must accept that our framework is not complete. There is a sense that we can
one day completely understand the universe, but this sense is misleading. The
truth is more that, for every testable physical goal, we may one day achieve
that goal; this is different from the idea that for every question, we may one
day have an answer. And the difference between these two is grounded in the
sense that any question we ask may hide within it ambiguity.

Am I claiming that every question/answer pair is necessarily ambiguous? I don't
quite think so. I think, for example, that well-defined math questions are not
ambiguous. But I think that, given any question we have not carefully considered
yet, it may contain an ambiguity. Something like: if it is a new question, then
there may be hidden ambiguity.

[By the way, for myself: I don't think I should prerequisite my writing of this
on any reading at all, including of the Wolfram article, {\em Why Does the
Universe Exist? Some Perspectives from Our Physics Project}. I believe I have
enough background already for this article, and my challenge now is to solidify
and communicate the ideas effectively.]

\bigskip
\hrule
\bigskip

I'm having a thought that maybe some questions don't fully make sense without
understanding the set of possible answers. This is a vague thought right now. I
remember I had a much older thought that a question is a set of answers, and now
I'm interesting in exploring that idea a little bit more.

I'll consider a few examples, and then I plan to put together a rough outline
for the article.

One theme below is that I'm looking at {\em why} questions. I think these are
interesting because some types of questions kind of build-in a set of answers.
For example, ``how much does $x$ cost?'' or ``how many $y$ ...'' or ``when ...''
{\em What} questions feel more flexibible, although I'm finding my intuition
seems a bit more inline with vagueness in terms of the {\em why} questions I've
thought of so far.

\begin{itemize}
    \item Why does anything exist at all? --- I think this is interesting
        because I wonder: what will we do with an answer? If the answer is that
        we might be in a simulation, then maybe we can test or verify that, or
        try to escape. If the answer is that the existence of the universe is
        fragile, then we may want to work to sustain the universe. It is, after
        all, where I keep all my stuff.
    \item Why is 2 the only even prime number? --- What I like here is the
        sequence of these three math questions because they show a clear
        progression of not-so-uncertain to much-more-uncertainty in terms of
        what a potential might even look like. In this first case, it feels like
        we want a simple proof that this is true, especially a proof that gives
        a feeling of intuition that the conclusion is naturally inevitable. In
        this case, I'd say that even numbers are basically defined as those
        numbers which are divisible by 2, so immediately anything larger than 2
        and even cannot be prime.
    \item Why isn't 1 a prime number? --- This one is trickier because 1 is not
        a prime not because of an obviously natural definition, but more because
        mathematicians have come to a general agreement about this edge case. So
        the question is, perhaps surprisingly the first time you hear of it,
        about history and context, and not as much about math itself. By the
        way, I think the main answer is that it's incredibly convenient to work
        with unique prime decompositions of all positive integers, and this only
        works out if you declare that 1 is not prime.
    \item Let $\phi=(1+\sqrt 5)/2.$ Why is $\phi^{50}$ almost an integer? --- If
        you haven't seen this question before, it probably looks confusing.
        It may look like ``Why is 23.000001 close to 23?'' which seems like an
        arbitrary and almost silly question. How could you answer that? And I
        think it's very interesting that the question can feel like this. To me,
        it's another piece of evidence that we cannot even grasp what is
        irrelevant to us. The concepts outside of our set of goals simply {\em
        cannot} exist to our minds, and this a profound boundary. Now, to answer
        this {\em why} question, (and keeping this brief for my self-aimed
        notes), there is a set of numbers which obey a kind of polynomial
        equation, and when such a kind of equation is obeyed, then an element in
        this set has the consequent property that powers of it become closer and
        closer to integers. This is not an obvious property of these numbers and
        it requires a little mathematical focus and thinking to follow the chain
        of reasoning from one point to the other. So it is the kind of answer
        where it's much easier to first understand the answer, and then later to
        realize you now can meaningfully answer this question, than (in my
        opinion) it is to think of the question and then come up with the
        answer.
\end{itemize}

\bigskip
\hrule
\bigskip

Ok, let's look at super rough outline draft for the aritcle.

[My main goal is to argue that truth itself is an invented concept, and
inherently vague in some ways. I really struggle to articulate this clearly. Can
we not rely on truth at all? I think we can in the way we rely upon a center of
gravity. It is useful and practical, but it's also good to remember that it is a
convenient abstraction and that there is something deeper beneath it. What is
beneath truth? The idea of truth only makes sense to us where we care about
things, and where there is ignorance. So, I'd say that (a) truth can only exist
in a mind; and (b) for us, beneath truth are our goals. \P\ \ Philosophy has
positioned itself as asking things that try to be more about the world than
about humanity. It's often bridged the gap between these, such as asking about
souls, or about consciousness, or about moral behaviors. But typically
when philosophers talk about any of those things, they speak as if talking about
consciousness is like a law of physics -- something abou the universe -- rather
than that consciousness is something about humans. And part of my thinking is
that we can learn more, and be more honest, if we acknowledge that all of these
ideas are necessarily based in our personal perspective. For example, when we
try to figure out if souls exist, we are dealing with a pernicious fiction that
we have invented ourselves. I find it easy to imagine aliens who would have a
great deal of trouble even understanding what a soul could possibly be, even as
a fiction. Another example is our attempt to understand consciousness, or to
decide what actions are good or bad. In all of these cases, we are fundamentally
speaking about the human experience, and it's folly to pretend that the ultimate
answer for us personally (meaning for humans) is the same as the ultimate answer
for the universe.]

I'm interested in trying to format this as mostly a straight article, but also
including counterpoint sections which attempt to sincerely capture a sense that
something has gone wrong with our line of thinking.

This might be a good approximate line of thinking:
\begin{enumerate}
    \item Phrase the question and the hypothesis. What is truth; it is invented
        and arbitrary. Motivate the question: How can we know we're doing a good
        job at answering questions that are to know if we've answered correctly.
        For example, what if we want to modify the scientific process at large;
        ie what is beyond $p-$values? How can we answer large-scale
        philosophical questions? How can we make better studies about how to
        decide on courses of action, such as social policies or health
        protocols?
    \item Give a brief basis for what truth is traditionally considered. Argue
        that this is not much of an answer.
    \item Introduce the different kinds of truth outlined above.
    \item Introduce the idea of a concept being invented versus an inherent part
        of the world. Such a center of gravity versus the fundamental particles
        of the universe.
    \item Introduce the idea of arbitrariness. For example, the fact that we
        read left-to-right (in English), or that we drive on a certain side of
        the road. Argue that invented ideas tend to be tied to arbitrariness,
        and that they often come with unsolved edge cases. I'd say, if we treat
        this mathematically, an invented concept can either be a full
        replacement for the real idea, or it is lossy. It's almost always lossy,
        and when it is lossy, we are glossing over edge cases.
    \item Argue that concepts only make sense to humans in terms of goals,
        questions, and answers. Within this framework is the idea that concepts
        require choices to have meaning. Without choice, there is no value in
        the concept. I'll go so far as to argue that we cannot even truly
        conceive of an idea without a choice for that idea, an opposition or
        alternative.
    \item Argue that our concept of truth is invented. We invented falseness,
        and truth only exists in opposition to falseness.
    \item Argue that our concept of truth is arbitrary. Most of our concepts of
        truth have unclear edge cases. That is most of what we deal with in our
        daily lives, and most of what we mean when we try to understand what is
        true. A small subset of our lives concern verifiable truths, and even
        then we have edge cases because we can easily discover that our concepts
        contain within them inherent mistakes. For example, we implicitly assume
        that the laws of physics exist, and that they do not change.
    \item Argue briefly for the psychological perspective of philosophy, as per
        my square bracketed notes above.
    \item Close with the key ideas from my opening paragraph in the square
        bracketed notes above.
\end{enumerate}


\end{document}  

























