\documentclass[11pt, oneside]{article}   	% use "amsart" instead of "article" for AMSLaTeX format
\usepackage{geometry}                		% See geometry.pdf to learn the layout options. There are lots.
\geometry{letterpaper}                   		% ... or a4paper or a5paper or ... 
%\geometry{landscape}                		% Activate for rotated page geometry
%\usepackage[parfill]{parskip}    		% Activate to begin paragraphs with an empty line rather than an indent
\usepackage{graphicx}				% Use pdf, png, jpg, or eps§ with pdflatex; use eps in DVI mode
								% TeX will automatically convert eps --> pdf in pdflatex		
\usepackage{amsmath}
\usepackage{amssymb}
\usepackage{float}
\usepackage{hyperref}
\usepackage{wrapfig}
\usepackage{refcount}
\usepackage{gensymb}

\hypersetup{
    colorlinks=true,
    linkcolor=black,   
    urlcolor=cyan,
}

\newtheorem{innercustomthm}{Answer}
\newenvironment{answer}[1]
  {\renewcommand\theinnercustomthm{#1}\innercustomthm}
  {\endinnercustomthm}

\newtheorem{thm}{Theorem}
\newtheorem{conj}{Conjecture}
\newtheorem{question}{Question}
\newtheorem{idea}{Idea}
\newtheorem{postidea}{Post Idea}
\newtheorem{defn}{Definition}
\newtheorem{ans}{Answer}
\newtheorem{prop}{Property}
\newtheorem{propn}{Proposition}
\newtheorem{claim}{Claim}
\newtheorem{obs}{Observation}

\newcommand{\up}[1]{\ensuremath{^\text{#1}}}

\newcommand{\Q}{\mathbb{Q}}
\newcommand{\R}{\mathbb{R}}
\newcommand{\D}{\mathbb{D}}
\newcommand{\Z}{\mathbb{Z}}

\newcommand{\spacer}[1]{\rule[-#1]{0pt}{#1}}

\newcommand{\qed}{\ensuremath{\Box}}
\newcommand\bpf[1][]{\smallskip\noindent{\bf Proof#1.}\quad}
\newcommand\epf{\qed\medskip}
\newcommand\hr{\bigskip\hrule\bigskip}

\begin{document}

{\bf Truth Is Not the Bedrock of Human Knowledge}

Tyler Neylon

\bigskip

This article explores the question:
\begin{question}\label{q1}
    What is truth?
\end{question}

I'll tell you why this question is useful for any explorer of knowledge, and
I'll argue that our concept of truth is a human invention as opposed to an idea
that is intrinsic to the universe.
Moreover, the concept of truth has arbitrary aspects to it, meaning that some of
the things we consider to be true are less elegant and more anthropocentric than
we may first realize. This has profound ramifications on how we, humans, ought
to view both ourselves and the world.

\section{The Box We'll Peek Inside}

A lot of philosophy is poorly motivated, so I'll start by
looking at what useful results can come out of this exploration.

As a kid, I was taught {\em the scientific method} as a way to
learn about the world -- to learn what is true.
I'll roughly summarize the scientific method as the statement
of a hypothesis, along with an alternative to test against, and
the collection and analysis of evidence to distinguish between
the two.

As I grew up, I
learned to question the completeness of this method.
I learned to appreciate the formality of a sound
mathematical proof, for example. A formal proof makes evidence collection seem
crude in
comparison. A logical proof is all-encompassing: the hypothesis -- now a theorem
-- is
precise and final,
invulnerable to the need for later revision.
One pitfall of the scientific method is the realization that it
is a process of guessing, collecting imperfect data, and doing our best to
connect that noisy data within our collection of guesses.

Beyond the awareness that the scientific method is noisy and imperfect, we can
begin to see that some of our questions do not fit cleanly within the confines
of its form. For example, we may find two conflicting theories of physics which
each partially explain the relevant observations. Then we must decide
which theory we believe. One might argue that the holes in one theory are worse
than the holes in another. One might argue that one theory is more elegant than
another.
These conversations aren't well captured by what we traditionally call the
scientific method, yet
they're common occurrences in knowledge-expanding communities,
whether the field is physics, computer science, or even, giving
ourselves some wiggle room, among chefs exploring theories of gastronomical
excellence.

All of this leads to the big question:

\begin{question}\label{q2}
    How can we learn what is true?
\end{question}

Question \ref{q2} can help us evolve how we learn.
It helps us to better understand and to build on what we call our
scientific method.
It may even
be the ultimate practical question -- this is the motivation I promised.

To connect the dots: When you open the door toward answering {\em how can we
learn what is true?}, you find the
Pandora's box that is {\em what is truth?} This is the box we'll open.

\section{Definitions of Truth}

\subsection{Correspondence Truth}

Truth is a concept so fundamental to human thinking that it's elusive to define
in simpler terms.
Perhaps the most traditional approach has been an idea called
{\em correspondence theory}\/:

\begin{defn}[Correspondence Truth]\label{d1}
    An idea is true when it corresponds to reality.
\end{defn}

I don't think this is a good definition.

Pretend we're defining the mathematical idea of a {\em number}, and we said
``a number is an element of the set of numbers.'' This 
has a lot in common with the correspondence definition of truth.
Specifically, this definition:
\begin{itemize}
    \item Relies on another concept ({\em the set of numbers})
        which is more complicated than the thing
        being defined. 
    \item Doesn't add much intuition about the thing being defined.
    \item Isn't easily testable. That is, we don't have a nice
        way to test if a thing is a number or not, in the context of not yet
        knowing a lot about the {\em set of numbers}.
\end{itemize}

On a metalevel, there's something interesting about our journey for a definition
of truth --- we're
looking for the definition of an idea that we already intuitively understand.
How will we know when we've succeeded? Abstractly, when the definition
appeals to our intuition.
If we think of examples that meet or fail to meet our intuition --- in our case,
true or false concepts --- then those examples ought to fit well with our
proposed definition.

% XXX Do I want to include this paragraph? It's a little digression from the
%     main line of thought.
This process of finding a definition is a case of
knowledge discovery which does not fit well with the traditional scientific
method --- although we could, with some creativity, treat concept examples as
experiments or observations, and consider a definition as a hypothesis.

With that in mind, let's consider an interesting claim, something that we
think of as a candidate for being true:
\begin{claim}\label{c1}
    If Plato were alive today, he would like pizza more than sushi.
\end{claim}
In this article, I'll use the words ``idea,'' ``concept,'' and ``claim'' as
synonyms for things that could be true or false. Some philosophers have thought
carefully about exactly what kind of thing can be true or false, but that isn't
the
focus here. I like the words I've chosen (idea/concept/claim) because none of
them strongly imply that what's being said must be correct; it's reasonable to
call ``1+1=3'' a {\em claim} or {\em idea}, and to see that it's false.

The correspondence definition of truth fails to shed light
on the truth or falsity of claim \ref{c1}.
The fact is that Plato isn't alive today, and --- for all practical purposes
--- we have no way to determine what kinds of modern food he would most like.
What we could do is make educated guesses, and try to convince each other
that one of these guesses is more likely to be correct.
This method is only loosely connected with an examination of reality, such as by
discussing historical evidence of ancient Greek diets.

With claim \ref{c1} in mind, I'm not saying that truth is not a correspondence
with reality.
Rather, I'm saying that we might find an improved definition of truth if we take
a careful look at how we decide something is true or false.
This is analogous to complaining that a definition such as ``a number is an
element of the set of numbers'' can be correct but unhelpful.

Learning from the weakness of definition \ref{d1}, a good definition:
\begin{itemize}
    \item Relies on prior concepts.
    \item Adds intuition.
    \item Helps test if something fits the definition.
\end{itemize}

% TODO Add a sentence or two to better tie together this direction with the next
%      section.

\subsection{Different Kinds of Truths}

When we start to look at {\em how} we decide something is true, we see a number
of different methods. In this section I'll describe kinds of truths classified
in this way. These definitions tend to be more intuitively useful and easier to
test than the correspondence definition above.

As a mathematician, I'd like to start with:

\begin{defn}[Mathematical Truth]
    A mathematical idea is true when we can provide a logically sound proof for
    it.
\end{defn}

As a brief reminder, a {\em sound argument} is a series of logical statements in
which the conclusion must logically follow from the premises, and in which the
premises are true in the context of the statement being proven.

The math-loving part of me would like to say that a proven mathematical idea
achieves an ideal level of truthiness. But in practice, this isn't quite
right. There are a number of reasons why mathematical ideas don't typically
achieve ``perfect truth:''
\begin{itemize}
    \item Math relies fundamentally on axioms being true, but those axioms are
        not proven. They are presented as self-evident, and in practice they
        have subtle and non-trivial consequences.
        One example is a geometric axiom called Euclid's parallel postulate,
        which states that, given a line and a point, there is a unique line
        through the point parallel to the line. If we assume this is true, then
        we can arrive at laws of geometry on a flat surface. If we assume this
        is false, then we can arrive at {\em equally valid} laws of geometry on
        non-flat surfaces, such as the surface of a sphere. The point here is to
        show that, as much as we'd like axioms to be self-evident, they aren't
        always so.
    \item Virtually all mathematical proofs are not fully formal
        arguments. That is, math literature is written for human consumption,
        and consists of largely natural language persuasion, as opposed to a
        computationally-verifiable proof. In practice, we sometimes find
        mistakes or omissions in these proofs, and they can be quite subtle. For
        example, mathematicians have famously disagreed about the correctness of
        Camille Jordan's original proof of the Jordan curve theorem.
        When experts disagree about the correctness of a proof, it shows that it
        can be quite difficult to see whether a proof is truly correct!
    \item Finally, even the rules of logic themselves are subject to debate. If
        we truly want to assume nothing, then it would be good to {\em know} we
        are using the correct rules of logic, rather than to assume them.
        You may feel secure that our time-tested rules of logic make sense ---
        but there's good reason to question some proof techniques.

        I'll focus, as an example, on proof by contradiction.
        A proof by contradiction
        only makes sense if we know the axioms being used do not contradict each
        other. When we have non-contradicting axioms, they're called {\em
        consistent}.

        Here's the catch: mathematicians can't prove that some key axioms are
        consistent --- at least not until they rely on {\em new axioms} for such
        a proof. But when you add a new axiom, you no longer know that the
        larger set of axioms is consistent! What we'd love is a single set of
        axioms $A$ where we can prove, using only the axioms of $A$, that our
        axioms are consistent. That way we can simultaneously believe all the
        proofs of $A$, as
        well as the proof that $A$ is consistent. Unfortunately, this is
        logically impossible for modern number
        systems.\footnote{For
        the curious: I'm referring to G\"odel's second
        incompleteness theorem as applied to Peano arithmetic.}
\end{itemize}

Surprisingly,
even when we strive for an ideal form of
certainty in the truth of a statement, there's a great deal of uncertainty.
There is subjectivity in our choice of axioms, in our belief in rules of logic,
and in our assessment of the correctness of an argument.
Most mathematicians perceive a well-known proof as complete and unassailable,
but the reality is that a human-written and human-read proof (vs
a computationally-verified proof) is just as much a natural language argument as
is the closing statement of a laywer in a courtroom.
The only difference is that the audience has a higher --- but still ultimately
subjective --- standard for what will convince them.

This is a theme I'm asking you to notice as we examine different kinds of
truth: That there is uncertainty everywhere we look. I'll phrase this as:

\begin{obs}
    For every concept presented as a truth, there is a corresponding reason
    given to believe in the concept, such as a mathematical proof, a reference
    to a scientific experiment, or an appeal based on expertise.

    When we examine these reasons, we find uncertainty.
\end{obs}

I'm not saying the ideas are always incorrect. I am saying that when we think we
{\em know} something is true, it's more honest to say that we {\em guess} it is
true. I'll come back to this high level after looking at a few more kinds of
truth.

If math is the most pure form of {\em abstract} truth I can imagine, then the
most pure form of {\em practical} truth must be concepts from physics that are
based on repeatable experiments:

\begin{defn}[Verifiable Truth]
    A verifiable idea is true when we can take an action whose outcome can
    distinguish between the truth or falsity of the idea.
\end{defn}

As an example, consider:
\begin{claim}
    Water boils at 100\/\degree C.
\end{claim}
We can boil some water and measure its temperature to test this.
Of course,
if we were to try this experiment in a setting with high or low air pressure, we
would find the boiling point to be slightly different. So it turns out that
there are other variables to account for in repeating an experiment ---
variables that we may not be aware of.

In considering mathematical truth, we found many sources of uncertainty.
Do all verifiable truths also contain uncertainty?

Let's try to imagine a highly specific verifiable claim, one where we can
account for all the relevant variables. If we can find such a claim, then we
could get a result that would always, without exception, be completely
consistent with the claim. If the result is based purely in logic, then we are
back to our concerns with mathematical truth, so this is only a new idea if we
think about physical experiments.

Let's imagine that one day we completely understand the laws of physics.
Further, let's imagine that there turn out to be quite simple rules, and that
all of the physical principles we use in practice, such as Newton's laws of
mechanics, are emergent properties of the simple rules. Suppose, for example,
that we lived in a universe based on Conway's game of life. This is a 2d world
of binary ``cells,'' each one being on or off in a given time step; each time
step is discrete (time moves forward frame by frame, as opposed to
continuously), and there is a set of rules to determine each time step from the
previous one.

% TODO XXX Clarify that I'll use the word "claim" for truth-bearers.

In such a world, we can make an extremely strong physics-based claim, such as:
\begin{claim}
    Cell $x$ will remain on from one time step to the next when 2 or 3
    neighboring cells are on in the previous time step.
\end{claim}

How could such a claim have any uncertainty at all to it?

Unfortunately, it can. Even if we do one day discover laws of physics that
explain every single experimental result or observation throughout all of human
experience, we're still making two fundamental assumptions:
\begin{itemize}
    \item We're assuming that any laws of physics exist at all.
    \item We're assuming not only that laws of physics exist, but that they
        remain unchanged throughout all of time.
\end{itemize}
As much as we may {\em believe} these assumptions are true, we do not truly {\em
know}, with certainty, that they must be true.

However much we understand the present, we know nothing about
the future with certainty.

The idea of verifiable truth is so general that we could consider using it as
{\em the} definition of truth. I see the appeal of this, but I don't think it's
quite right because it glosses over other kinds of truth that I'll talk
about. For example, mathematicans simply do not accept a theorem to be true no
matter how many times you test it and find it to be true --- they require
something beyond repeated experimentation. And there are other interesting
concepts which we call true, and which do not comfortably fit under the umbrella
of verifiability. Consider, for example, the claim:
\begin{claim}\label{c2}
    Han shot first.
\end{claim}
This is in reference to Han Solo's encounter with the bounty hunter Greedo in
Mos Eisley, as depicted in the 1977 film {\em Star Wars}.
In the original version of the film, Han Solo clearly shot Greedo before Greedo
had a chance to
fire at Han, but in later, edited, versions of the film, Greedo either shot at
Han first or nearly simultaneously. So the truth of claim \ref{c2} is not
obvious.

I don't think claim \ref{c2} is verifiable in the same way as the claim that
water boils at 100\degree C.
It is about a fictional event in a widely-known story.
For many narrative-based questions, we could simply ask the author since they
have a kind of authority over what ``really happened'' in the story they
created. But in this case, the story of {\em Star Wars} has effectively
graduated to a level of American mythology, in which case the authorship of a
single living person (George Lucas, in this case) is less meaningful.
Furthermore, the owner of the rights to the film, Disney, is no longer strongly
associated with the original author.

We're examining a new kind of truth:
\begin{defn}[Authoritative Truth]
    A narrative idea is true when the party recognized as the authority on the
    narrative claims the idea to be correct. These ideas are understood to be
    about fictional stories.
\end{defn}

We could ask if Harry Potter likes cilantro. This question is not addressed in
any of the writings of author J.K.~Rowling. However, if she were to publicly
declare the answer one way or another, it would be accepted as canonically true.
Just as much as we accept the statement ``Darth Vader is Luke's father,'' we
would also accept ``Harry Potter loathes cilantro.''

This kind of truth may feel less real,
but it's still one that we discuss and
care about. If a kid asked you where Santa Claus lives, you would tell them he
lives at the North Pole. There is a shared narrative here; if you were to say he
lives at the South Pole, this would feel incorrect.

While fictional stories are not about things that happened in reality, when we
talk about these stories, we're still talking about actual events. The telling
of the story, the listening of the story, and our discussion and thoughts of the
story are all real events.
If the story has enough appeal,
then it becomes something greater than a single telling of the story. Stories,
to humans, are a part of our experience and our learning in terms of what can
happen, how people behave, and how we learn about life. Our own thoughts and
actions evolve as we experience narratives, whether they are our personal
experiences, those we know of our community, or those stories we hear
indirectly.

For example, we generally do not learn that murder is bad by committing or
witnessing murder. Instead, we learn about it as we grow. In some cases, we may
hear stories of loss, and at some point, vicariously experience some pain of
loss that helps us to appreciate the harm of taking a life.

To me, moral ideas never feel black and white, but so many people consider them
to be such that I'll include this kind of truth:
\begin{defn}[Moral Truth]
    A moral guideline is true when a society following it is better off than a
    society that ignores it.
\end{defn}

There's a lot in that definition. I've chosen to work with a variation of
Immanuel Kant's categorical imperative --- his style of the golden rule. It's
quite a rabbit hole to consider this definition carefully, so for now I'll ask
you to accept that we're not focusing on morality, but rather noticing that if I
say ``murder is wrong,'' this is an idea which we may say is true or false, and
which is more clearly a {\em moral truth} than the other kinds of truth in this
article.

There is yet another kind of truth adjacent to morality. Consider the claim:
\begin{claim}
    Forks go on the left side of your plate.
\end{claim}

There is a sense of agreement about this claim, and yet it is clearly not a
result of a logical argument, nor of a physical experiment. Neither is it
a moral directive since no one suffers
if a fork is placed on the right of a plate. There are other truths
not far from this kind, such as the fact that Americans drive on the right side
of the road, while British drivers stay on the left.

The interesting thing here is that decisions have been made where the important
result is not that we chose the {\em correct} outcome, such as driving on the
left or right side of the road, but rather that we are {\em consistent} about
the result. In the case of driving, consistency provides safety; the case of
utensil etiquette, the result seems to be an expression of cultural status or
awareness.
In either case, the test for correctness is an understanding of what most people
tend to do. In other words, the truth of these things is based on noticing what
the majority already treats as the correct decision:
\begin{defn}[Democratic Truth]
    A social convention is true when the vast majority of a society consider it
    to be true.
\end{defn}

It's not always obvious how democratic truths form.
Did a monarch one day
see a fork on the right of her plate, felt funny about it, and decapitate her
table-setting staff members? And from that day forth, forks were carefully
placed on the left?

This possible fork-on-the-left origin story is an example of an idea being
decided by a single individual.
In some rare cases, it does seem as if a single point of history determines a
convention.
Consider the sandwich.
The word {\em sandwich}, referring to food, is distinctly traceable to John
Montagu, the $4^\text{th}$ Earl of Sandwich. The story, provided by plausible
historical accounts, is that he liked to eat quickly and cleanly while either
gambling or working (depending on your source), and a bread-enclosed meal did
the trick.

More often, it appears that conventions evolve slowly. Rather than arising from
a single conscious decision, they appear to be an accumulation of natural
smaller steps, with a reason behind each step.
These might not always be {\em good} reasons,
but
when you examine the history of a concept,
you often find more sense
than you might expect. Newts were once ewts; when people saw one they would tell
their friend they saw ``an ewt,'' which was so easy to confuse with ``a newt''
that the word changed. People once considered the Mediterranean Sea to be the
sea in the middle (medi) of the earth (terran).

In the examples above, we're seeing ideas, like words, that live in multitudes
among communities --- such as French words in France, or mathematical lingo
among mathematicians. These ideas shift and adapt as the world changes.
Some ideas receive more attention while others, apparently less useful, dwindle.
This brings us to another perspective on truth that's worth
consideration in its own right:
\begin{defn}[Evolutionary Truth]
    A communal concept is true when the persistence of that concept corresponds
    with the persistence of the community.
\end{defn}

Something peculiar about this idea of truth is that it does not directly
convince us that the concept corresponds to reality, and this may irk your
intuition. Yet every good idea does meet this definition.
For example, if one community believes in germs, and another
doesn't, then over time the germ-believing community is more likely to survive,
and the idea of germs with them.
Ideas are the social genes of the community.
They mutate and change over time, and the theory of natural selection
applies to the ideas just as they do to traditional genes in a species.

At the same time that good ideas tend to be evolutionary truths, there's room
for other ideas to tag along for the ride.
While good ideas are evolutionary truths because they promote survival, other
ideas may count as evolutionary truths simply because they are good at
keeping themselves alive, rather than keeping the community alive.
Some beliefs of organized religion seem to fit into this category.
Religion historically helped communities
by encouraging them to work together, to help each other
out, and to remain organized. Those are genuine benefits. Along with those
benefits came ideas that do not correspond with reality, nor do they confer
verifiability or utility in practical decisions. For example, the belief that
Zeus exists as a god, and is the son of the titans Cronus and Rhea.

You might say that these religious beliefs are authoritative truths, but people
who see such ideas as real would disagree; authoritative truths are recognized
as fictional.

Just as there is uncertainty in previous kinds of truth, there is uncertainty
here. Indeed, there are many incompatible religious beliefs, so they simply
cannot all be correct at the same time.

As we work through these kinds of truth, it's tempting to ask:
\begin{question}\label{q1}
    Is there a single all-encompassing definition for truth?
\end{question}
In other words, is one of our definitions the {\em main} definition, with the
others describing subsets of truths?
Just as we can provide multiple definitions of an English word, I think there
is more than one valid definition of truth. Even in mathematics, we may find
that we can define a formal and technical concept in different ways, and those
ways end up being equivalent after some analysis.
Accordingly, I'll provide more than one answer to question \ref{q1}.

First I'll argue that:
\begin{answer}{3a.}
    When we seek to describe the truths that groups of people tend to believe,
    then we're talking about evolutionary truth.
\end{answer}

This answer is almost tautological in the sense that I'm aligning the context of
the answer with the definition of evolutionary truth. I still think there's
an interesting thought here: we're taking the vast wealth of thought about
biological evolution and seeing that it applies to what we accept as true. This
is mostly an old idea in the sense that biologists such as W.D.~Hamilton and
Richard Dawkins have previously considered evolution as applying to ideas in
addition to the traditional biological settings.

Based on this perspective, humans
may deserve less credit than we tend to give ourselves for having many
brilliant ideas. Just as evolution has ``invented'' life, flight, eyeballs and
brains without individual insights, perhaps the simple mechanism of billions of
people guessing-and-checking many possibilities deserves some
credit for human innovation.

While I think evolutionary truth captures the {\em descriptive} idea of
what ends up getting believed, I think it is not a {\em normative} idea of what
we intuitively feel is true. In other words, evolutionary truth is talking more
about how ideas survive than about what an all-knowing being would agree with.
And our intuition for what's true is closer to this all-knowing thought
experiment.

This brings me to the last kind of truth I'd like to describe:
\begin{defn}[Effective Truth]
    A useful idea is true if someone using it tends to achieve their goals by
    doing so.
\end{defn}

For example, suppose you believe that red bowling balls are luckier than any
other color. It just so happens that your red bowling ball fits your hand better
than others that are available to you. In this case, your belief helps you to
achieve higher bowling scores, so it is effectively true.
I've chosen this example to pique your sense of imperfection, but next I'll
argue that all of our ideas are like this; that is, I don't think the
imperfection is in this definition, but is a necessary property of truth itself.

Consider the simple and useful formula:
\begin{claim}[Newton's second law]
    $F = m \cdot a.$
\end{claim}

Unlike your belief that red bowling balls are lucky, this idea probably feels
unsuperstitious and reasonable. Here's the thing: it's not true.
Newton's laws only apply to non-tiny
objects that are moving at what you might call non-relativistic (not too fast)
speeds. Even in those cases where they do apply,
it turns out Newton's laws are only
approximations that tend to be close to the truth.

How is Newton's second law really different from the idea that red bowling balls
are
lucky? There is a difference in that, if we were to scientifically explore both,
we'd find one of them is, as an approximation, correct much more often than the
other. Can we say that one is true and the other is false? I don't think so;
both will fail to be perfectly true in many experiments.

What we end up with instead is a way to evaluate different shades of truthiness.
We also find that this kind of truth does not live in a vacuum; it makes
sense only in the context of goals. A ``very true'' useful idea, in the context
of a certain goal, will achieve that goal almost every time. If a goal can
almost never be achieved, then we may still care about ``slightly true'' ideas
---
ones which achieve their goal only a small percentage of the time.










% TODO rest of the section:
% evolutionary truth, effective truth, they are all uncertain, they can easily
% overlap, I see effective and evolutionary truths as the key ideas, depending
% on if you identify as an individual or as a group; and that verifiable tends
% to be similar to effective, but that verifiability pretends we don't have
% goals, and I think it's more honest to understand that we always have goals.
% so we start with this idea that truth is objective and exists independently of
% ourselves. next we'll ask how this could possibly not be true


% Some metathoughts
% 1) How does all of this relate to realism vs anti-realism?
% 2) How does all of this relate to cogito ergo sum?
%
% I suspect there's more nuance than this, but we might simplify and say that
% Descartes was being somewhat anti-realist in saying that he couldn't know
% anything for sure; the exception of saying that he exists is like an extremely
% limited kind of realism. Instead of saying the world exists, he only knows
% that he exists.
%
% My main claim is that the idea of truth is invented. I don't think this needs
% to be a strong statement about realism or anti-realism. If the world exists,
% then I'm saying truth is something we add to that world in our minds. If we
% don't think of the world as necessarily existing independently, then we still
% have some idea of the world (perhaps as necessarily uncertain, or
% perspective-dependent), and still within that world truth is an invention.
% I'm not sure that I need to say all of this, but I'm not 100% decided (leaning
% toward no). If I mention anything, maybe I would say that there is a
% difference between Descartes and I: Descartes was saying that we always have
% uncertainty about what is real outside of ourselves, and I'm saying that the
% idea of truth only makes sense because of our own intrinsic uncertainty (what
% I plan to call ignorance).
%

% State, either before or after this math part, what I'm getting at in the
% big picture. Something like we think these things are true, but we don't have
% as much certainty as we'd like.

% Notice these things: We act as if we know mathematical ideas to be certainly
% correct, but we don't really. I'm not saying they're all wrong, but rather,
% that, especially as the ideas become more intricate, that we could easily be
% making mistakes along the way, or that we could be making assumptions without
% fully appreciating the implications of the assumptions we've made. Math is not
% black-and-white. It has fundamentally subjective components --- always.






%
% ______________________________________________________________________
% Some big-picture notes.
%

% Where are we going with the big picture of this section?
%  The next couple sections are: defining invention and arbitrariness.
% So it would be good to start mentioning properties that will cover or lead to
% those two things.

% Some notes focusing on showing that truth is an invented/arbitrary idea:
% * Falseness is more clearly an invention. Truth only exists in opposition to
%   falseness; therefore truth is just as invented as falseness. I'm using the
%   word "invention" a little differently from the most traditional English
%   usage of the word here, and it's worth trying to keep that clear.
% * Ideas only make sense in the presence of ignorance and goals. This
%   dependency shows that truth is not an intrinsic part of the universe.
%   I think I can be more articulate on this point. It also heavily overlaps
%   with the next point. I think that these two points combined: (a) have a core
%   argument, which is this point, and (b) have an intuition-building thought
%   experiment, which is the next point.
% * The know-everything perspective. If we knew everything, then questions no
%   longer make sense. For example, if blue where not blue, what color would it
%   be? This helps us to see, intuitively, that our ideas depend essentially on
%   our limitations. They cannot be intrinsic properties of the universe because
%   the universe does not have goals or questions.

%
% ______________________________________________________________________
%

\hr

Notes for section 2.

It's like saying a number is an element of the set of numbers.
It hasn't helped us understand anything, and it feels like it's
defined in terms of more complex ideas rather than in terms of
something simpler. It offers us little in the way of testability.

Metanote that we're doing something interesting in finding a philosophical
definition. It's like we're looking for a way to describe something that
we already know, as opposed to a definition in math literature, where, when
learning, the definition comes first (for the student), and the intuition often
follows after.

Segue to the next section? The next section will introduce different types of
truth. Which steak would Einstein have liked better? It's a weird kind of
question because it's hard to test, yet it feels like it still has a true
answer. I'm mentioning this example because I'm trying to show a difference
between corresponding to reality and the way we actually work with certain
claims.

What I'm moving toward here is that I'm aiming for properties that are nice to
have in a good definition.

It's nice if a definition adds intuition to the thing being defined. For
example, can we understand how this concept is useful, or how it historically
evolved. It's nice if the definition gives a way to test if something fits the
definition, and if the definition is built in terms of concepts that we can
understand before we understand the defined idea.

\hr


One way to try on definitions or theories is to see how they work with examples,
so in this article I'll provide examples of candidate truths --- things which
might count as true or false. I'll call these {\em claims}, although I am not
literally claiming these to be true.

\begin{claim}
    The party I'm throwing tonight begins at seven.
\end{claim}





\bigskip
\hrule
\bigskip

{\em MAH NOTES}

Dearest reader, if you see this section at all, then I, the author,
have made some sort of grave error in compiling and/or distributing this
article. I apologize profusely and ensure you that anything untoward said
about you or anyone you care about was surely and entirely in jest.

{\bf Motivation}

I want to be able to figure out if something is true or not.
Now, later on, I'll lean toward the idea of effective truth.
So I will need to say how this ends up being useful, but I'd rather save
that for a recap at the end.

I might want to say how it's not always obvious what's true.
Maybe that can be the first exploration of the article.

{\bf What kind of truths are tricky?}

We think of truth as the state of the world. But often it's not that simple.

For example, did Greedo shoot first? It feels like there should be an answer to
this, but it's not obvious how to decide this.

This sentence is false. Is it?

There are kind of ethical questions.
What makes privacy a fundamental right?
What is bad about talking about taboo subjects?
When we have a candidate answer to these, we want a way to test how true they
are.

Axioms of math. What makes something self-evident?

Newtonian mechanics. We know they it is a false model, but we still find it
useful.

{\bf Defining truth}

Here is an outline of ideas I can cover:

Ideas for different kinds of truth:
\begin{itemize}
    \item Verifiable truth; like a repeatable physics experiment.
    \item Authorative truth; like an author saying what a character did.
    \item Democratic truth; like a group deciding what's cool or polite.
    \item Effective truth; an idea that tends to produce a desired result.
    \item Evolutionary truth; an idea that tends to survive.
          Perhaps it's convincing, or perhaps it is self-reinforcing, like
          religion.
\end{itemize}

\bigskip
\hrule
\bigskip

I may want to take this article in a slightly different direction. I've already
kind of outlined many ideas about the nature of truth in an article called ``A
Model of Human Thought: Philosophy.'' But still, the focus of that article was
not actually truth, and I simply included it because I had not written those
ideas up previously, and I wanted them for a more solid context to model human
thought.

The slightly different direction here could be to build an intuition around the
idea that our concept of truth is a somewhat arbitrary and invented idea. A
practical grounding could be in the basis of wanting to know what is true as a
good scientist. And we get to some quite difficult questions such as, is this
axiom true? Why does anything exist? How do we think about abstract
contradictions such as this-sentence-is-false, or the idea of proof by
contradiction?

And one conclusion is that if we want to expand our boundaries of learning, then
we must accept that our framework is not complete. There is a sense that we can
one day completely understand the universe, but this sense is misleading. The
truth is more that, for every testable physical goal, we may one day achieve
that goal; this is different from the idea that for every question, we may one
day have an answer. And the difference between these two is grounded in the
sense that any question we ask may hide within it ambiguity.

Am I claiming that every question/answer pair is necessarily ambiguous? I don't
quite think so. I think, for example, that well-defined math questions are not
ambiguous. But I think that, given any question we have not carefully considered
yet, it may contain an ambiguity. Something like: if it is a new question, then
there may be hidden ambiguity.

[By the way, for myself: I don't think I should prerequisite my writing of this
on any reading at all, including of the Wolfram article, {\em Why Does the
Universe Exist? Some Perspectives from Our Physics Project}. I believe I have
enough background already for this article, and my challenge now is to solidify
and communicate the ideas effectively.]

\bigskip
\hrule
\bigskip

I'm having a thought that maybe some questions don't fully make sense without
understanding the set of possible answers. This is a vague thought right now. I
remember I had a much older thought that a question is a set of answers, and now
I'm interesting in exploring that idea a little bit more.

I'll consider a few examples, and then I plan to put together a rough outline
for the article.

One theme below is that I'm looking at {\em why} questions. I think these are
interesting because some types of questions kind of build-in a set of answers.
For example, ``how much does $x$ cost?'' or ``how many $y$ ...'' or ``when ...''
{\em What} questions feel more flexibible, although I'm finding my intuition
seems a bit more inline with vagueness in terms of the {\em why} questions I've
thought of so far.

\begin{itemize}
    \item Why does anything exist at all? --- I think this is interesting
        because I wonder: what will we do with an answer? If the answer is that
        we might be in a simulation, then maybe we can test or verify that, or
        try to escape. If the answer is that the existence of the universe is
        fragile, then we may want to work to sustain the universe. It is, after
        all, where I keep all my stuff.
    \item Why is 2 the only even prime number? --- What I like here is the
        sequence of these three math questions because they show a clear
        progression of not-so-uncertain to much-more-uncertainty in terms of
        what a potential might even look like. In this first case, it feels like
        we want a simple proof that this is true, especially a proof that gives
        a feeling of intuition that the conclusion is naturally inevitable. In
        this case, I'd say that even numbers are basically defined as those
        numbers which are divisible by 2, so immediately anything larger than 2
        and even cannot be prime.
    \item Why isn't 1 a prime number? --- This one is trickier because 1 is not
        a prime not because of an obviously natural definition, but more because
        mathematicians have come to a general agreement about this edge case. So
        the question is, perhaps surprisingly the first time you hear of it,
        about history and context, and not as much about math itself. By the
        way, I think the main answer is that it's incredibly convenient to work
        with unique prime decompositions of all positive integers, and this only
        works out if you declare that 1 is not prime.
    \item Let $\phi=(1+\sqrt 5)/2.$ Why is $\phi^{50}$ almost an integer? --- If
        you haven't seen this question before, it probably looks confusing.
        It may look like ``Why is 23.000001 close to 23?'' which seems like an
        arbitrary and almost silly question. How could you answer that? And I
        think it's very interesting that the question can feel like this. To me,
        it's another piece of evidence that we cannot even grasp what is
        irrelevant to us. The concepts outside of our set of goals simply {\em
        cannot} exist to our minds, and this a profound boundary. Now, to answer
        this {\em why} question, (and keeping this brief for my self-aimed
        notes), there is a set of numbers which obey a kind of polynomial
        equation, and when such a kind of equation is obeyed, then an element in
        this set has the consequent property that powers of it become closer and
        closer to integers. This is not an obvious property of these numbers and
        it requires a little mathematical focus and thinking to follow the chain
        of reasoning from one point to the other. So it is the kind of answer
        where it's much easier to first understand the answer, and then later to
        realize you now can meaningfully answer this question, than (in my
        opinion) it is to think of the question and then come up with the
        answer.
\end{itemize}

\bigskip
\hrule
\bigskip

Ok, let's look at super rough outline draft for the aritcle.

[My main goal is to argue that truth itself is an invented concept, and
inherently vague in some ways. I really struggle to articulate this clearly. Can
we not rely on truth at all? I think we can in the way we rely upon a center of
gravity. It is useful and practical, but it's also good to remember that it is a
convenient abstraction and that there is something deeper beneath it. What is
beneath truth? The idea of truth only makes sense to us where we care about
things, and where there is ignorance. So, I'd say that (a) truth can only exist
in a mind; and (b) for us, beneath truth are our goals. \P\ \ Philosophy has
positioned itself as asking things that try to be more about the world than
about humanity. It's often bridged the gap between these, such as asking about
souls, or about consciousness, or about moral behaviors. But typically
when philosophers talk about any of those things, they speak as if talking about
consciousness is like a law of physics -- something abou the universe -- rather
than that consciousness is something about humans. And part of my thinking is
that we can learn more, and be more honest, if we acknowledge that all of these
ideas are necessarily based in our personal perspective. For example, when we
try to figure out if souls exist, we are dealing with a pernicious fiction that
we have invented ourselves. I find it easy to imagine aliens who would have a
great deal of trouble even understanding what a soul could possibly be, even as
a fiction. Another example is our attempt to understand consciousness, or to
decide what actions are good or bad. In all of these cases, we are fundamentally
speaking about the human experience, and it's folly to pretend that the ultimate
answer for us personally (meaning for humans) is the same as the ultimate answer
for the universe.]

I'm interested in trying to format this as mostly a straight article, but also
including counterpoint sections which attempt to sincerely capture a sense that
something has gone wrong with our line of thinking.

This might be a good approximate line of thinking:
\begin{enumerate}
    \item Phrase the question and the hypothesis. What is truth; it is invented
        and arbitrary. Motivate the question: How can we know we're doing a good
        job at answering questions that are to know if we've answered correctly.
        For example, what if we want to modify the scientific process at large;
        ie what is beyond $p-$values? How can we answer large-scale
        philosophical questions? How can we make better studies about how to
        decide on courses of action, such as social policies or health
        protocols?
    \item Give a brief basis for what truth is traditionally considered. Argue
        that this is not much of an answer.
    \item Introduce the different kinds of truth outlined above.
    \item Introduce the idea of a concept being invented versus an inherent part
        of the world. Such a center of gravity versus the fundamental particles
        of the universe.
    \item Introduce the idea of arbitrariness. For example, the fact that we
        read left-to-right (in English), or that we drive on a certain side of
        the road. Argue that invented ideas tend to be tied to arbitrariness,
        and that they often come with unsolved edge cases. I'd say, if we treat
        this mathematically, an invented concept can either be a full
        replacement for the real idea, or it is lossy. It's almost always lossy,
        and when it is lossy, we are glossing over edge cases.
    \item Argue that concepts only make sense to humans in terms of goals,
        questions, and answers. Within this framework is the idea that concepts
        require choices to have meaning. Without choice, there is no value in
        the concept. I'll go so far as to argue that we cannot even truly
        conceive of an idea without a choice for that idea, an opposition or
        alternative.
    \item Argue that our concept of truth is invented. We invented falseness,
        and truth only exists in opposition to falseness.
    \item Concepts only make sense in the presence of ignorance and goals.
        What color is blue if it might not be blue? The nonsensicality of that
        kind of question helps us to see how dependent ideas are on our
        ignorance. All questions are like that, though we can't see it. Explain
        the know-everything thought experiment. If I want to differentiate from
        cogito ergo sum, this is a good place for that.
    \item Argue that our concept of truth is arbitrary. Most of our concepts of
        truth have unclear edge cases. That is most of what we deal with in our
        daily lives, and most of what we mean when we try to understand what is
        true. A small subset of our lives concern verifiable truths, and even
        then we have edge cases because we can easily discover that our concepts
        contain within them inherent mistakes. For example, we implicitly assume
        that the laws of physics exist, and that they do not change.
    \item Argue briefly for the psychological perspective of philosophy, as per
        my square bracketed notes above.
    \item Close with the key ideas from my opening paragraph in the square
        bracketed notes above.
\end{enumerate}


\end{document}  

























