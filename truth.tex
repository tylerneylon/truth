\documentclass[11pt, oneside]{article}   	% use "amsart" instead of "article" for AMSLaTeX format
\usepackage{geometry}                		% See geometry.pdf to learn the layout options. There are lots.
\geometry{letterpaper}                   		% ... or a4paper or a5paper or ... 
%\geometry{landscape}                		% Activate for rotated page geometry
%\usepackage[parfill]{parskip}    		% Activate to begin paragraphs with an empty line rather than an indent
\usepackage{graphicx}				% Use pdf, png, jpg, or eps§ with pdflatex; use eps in DVI mode
								% TeX will automatically convert eps --> pdf in pdflatex		
\usepackage{amsmath}
\usepackage{amssymb}
\usepackage{float}
\usepackage{hyperref}
\usepackage{wrapfig}
\usepackage{refcount}

\hypersetup{
    colorlinks=true,
    linkcolor=black,   
    urlcolor=cyan,
}

\newtheorem{thm}{Theorem}
\newtheorem{conj}{Conjecture}
\newtheorem{question}{Question}
\newtheorem{idea}{Idea}
\newtheorem{postidea}{Post Idea}
\newtheorem{defn}{Definition}
\newtheorem{ans}{Answer}
\newtheorem{prop}{Property}

\newcommand{\up}[1]{\ensuremath{^\text{#1}}}

\newcommand{\Q}{\mathbb{Q}}
\newcommand{\R}{\mathbb{R}}
\newcommand{\D}{\mathbb{D}}
\newcommand{\Z}{\mathbb{Z}}

\newcommand{\spacer}[1]{\rule[-#1]{0pt}{#1}}

\newcommand{\qed}{\ensuremath{\Box}}
\newcommand\bpf[1][]{\smallskip\noindent{\bf Proof#1.}\quad}
\newcommand\epf{\qed\medskip}

\begin{document}

{\bf A Poetic Title for This Article}

Tyler Neylon

\bigskip

This article explores the question: What is truth?

A lot of philosophy is poorly motivated, so I'll start by
looking at what useful results can come out of this exploration.

\bigskip
\hrule
\bigskip

{\em MAH NOTES}

Dearest reader, if you see this section at all, then I, the author,
have made some sort of grave error in compiling and/or distributing this
article. I apologize profusely and ensure you that anything untoward said
about you or anyone you care about was surely and entirely in jest.

{\bf Motivation}

I want to be able to figure out if something is true or not.
Now, later on, I'll lean toward the idea of effective truth.
So I will need to say how this ends up being useful, but I'd rather save
that for a recap at the end.

I might want to say how it's not always obvious what's true.
Maybe that can be the first exploration of the article.

{\bf What kind of truths are tricky?}

We think of truth as the state of the world. But often it's not that simple.

For example, did Greedo shoot first? It feels like there should be an answer to
this, but it's not obvious how to decide this.

This sentence is false. Is it?

There are kind of ethical questions.
What makes privacy a fundamental right?
What is bad about talking about taboo subjects?
When we have a candidate answer to these, we want a way to test how true they
are.

Axioms of math. What makes something self-evident?

Newtonian mechanics. We know they it is a false model, but we still find it
useful.

{\bf Defining truth}

Here is an outline of ideas I can cover:

Ideas for different kinds of truth:
\begin{itemize}
    \item Verifiable truth; like a repeatable physics experiment.
    \item Authorative truth; like an author saying what a character did.
    \item Democratic truth; like a group deciding what's cool or polite.
    \item Effective truth; an idea that tends to produce a desired result.
    \item Evolutionary truth; an idea that tends to survive.
          Perhaps it's convincing, or perhaps it is self-reinforcing, like
          religion.
\end{itemize}

\bigskip
\hrule
\bigskip

I may want to take this article in a slightly different direction. I've already
kind of outlined many ideas about the nature of truth in an article called ``A
Model of Human Thought: Philosophy.'' But still, the focus of that article was
not actually truth, and I simply included it because I had not written those
ideas up previously, and I wanted them for a more solid context to model human
thought.

The slightly different direction here could be to build an intuition around the
idea that our concept of truth is a somewhat arbitrary and invented idea. A
practical grounding could be in the basis of wanting to know what is true as a
good scientist. And we get to some quite difficult questions such as, is this
axiom true? Why does anything exist? How do we think about abstract
contradictions such as this-sentence-is-false, or the idea of proof by
contradiction?

And one conclusion is that if we want to expand our boundaries of learning, then
we must accept that our framework is not complete. There is a sense that we can
one day completely understand the universe, but this sense is misleading. The
truth is more that, for every testable physical goal, we may one day achieve
that goal; this is different from the idea that for every question, we may one
day have an answer. And the difference between these two is grounded in the
sense that any question we ask may hide within it ambiguity.

Am I claiming that every question/answer pair is necessarily ambiguous? I don't
quite think so. I think, for example, that well-defined math questions are not
ambiguous. But I think that, given any question we have not carefully considered
yet, it may contain an ambiguity. Something like: if it is a new question, then
there may be hidden ambiguity.

[By the way, for myself: I don't think I should prerequisite my writing of this
on any reading at all, including of the Wolfram article, {\em Why Does the
Universe Exist? Some Perspectives from Our Physics Project}. I believe I have
enough background already for this article, and my challenge now is to solidify
and communicate the ideas effectively.]

\bigskip
\hrule
\bigskip

I'm having a thought that maybe some questions don't fully make sense without
understanding the set of possible answers. This is a vague thought right now. I
remember I had a much older thought that a question is a set of answers, and now
I'm interesting in exploring that idea a little bit more.

I'll consider a few examples, and then I plan to put together a rough outline
for the article.

One theme below is that I'm looking at {\em why} questions. I think these are
interesting because some types of questions kind of build-in a set of answers.
For example, ``how much does $x$ cost?'' or ``how many $y$ ...'' or ``when ...''
{\em What} questions feel more flexibible, although I'm finding my intuition
seems a bit more inline with vagueness in terms of the {\em why} questions I've
thought of so far.

\begin{itemize}
    \item Why does anything exist at all? --- I think this is interesting
        because I wonder: what will we do with an answer? If the answer is that
        we might be in a simulation, then maybe we can test or verify that, or
        try to escape. If the answer is that the existence of the universe is
        fragile, then we may want to work to sustain the universe. It is, after
        all, where I keep all my stuff.
    \item Why is 2 the only even prime number? --- What I like here is the
        sequence of these three math questions because they show a clear
        progression of not-so-uncertain to much-more-uncertainty in terms of
        what a potential might even look like. In this first case, it feels like
        we want a simple proof that this is true, especially a proof that gives
        a feeling of intuition that the conclusion is naturally inevitable. In
        this case, I'd say that even numbers are basically defined as those
        numbers which are divisible by 2, so immediately anything larger than 2
        and even cannot be prime.
    \item Why isn't 1 a prime number? --- This one is trickier because 1 is not
        a prime not because of an obviously natural definition, but more because
        mathematicians have come to a general agreement about this edge case. So
        the question is, perhaps surprisingly the first time you hear of it,
        about history and context, and not as much about math itself. By the
        way, I think the main answer is that it's incredibly convenient to work
        with unique prime decompositions of all positive integers, and this only
        works out if you declare that 1 is not prime.
    \item Let $\phi=(1+\sqrt 5)/2.$ Why is $\phi^{50}$ almost an integer? --- If
        you haven't seen this question before, it probably looks confusing.
        It may look like ``Why is 23.000001 close to 23?'' which seems like an
        arbitrary and almost silly question. How could you answer that? And I
        think it's very interesting that the question can feel like this. To me,
        it's another piece of evidence that we cannot even grasp what is
        irrelevant to us. The concepts outside of our set of goals simply {\em
        cannot} exist to our minds, and this a profound boundary. Now, to answer
        this {\em why} question, (and keeping this brief for my self-aimed
        notes), there is a set of numbers which obey a kind of polynomial
        equation, and when such a kind of equation is obeyed, then an element in
        this set has the consequent property that powers of it become closer and
        closer to integers. This is not an obvious property of these numbers and
        it requires a little mathematical focus and thinking to follow the chain
        of reasoning from one point to the other. So it is the kind of answer
        where it's much easier to first understand the answer, and then later to
        realize you now can meaningfully answer this question, than (in my
        opinion) it is to think of the question and then come up with the
        answer.
\end{itemize}

\bigskip
\hrule
\bigskip

Ok, let's look at super rough outline draft for the aritcle.


\end{document}  

























